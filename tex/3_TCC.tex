Neste capítulo propomos novos métodos de seleção de cláusula para o KSP.

Fizemos um experimento inicial para averiguar o desempenho do KSP sobre o LWB com os algoritmos de seleção de cláusula já implementados. O sistema utilizado tem processador AMD FX-6300 com clock base de 3.5 GHz, 6 Gigabyte de memória RAM no Sistema Operacional Ubuntu 18.04. Escolhemos a versão 0.1.2 do KSP, a versão pública mais recente na data do experimento. Nesse experimento usamos 21 fórmulas satisfatíveis e 21 insatisfatíveis para cada família. Para cada fórmula foram dados 10 segundos de tempo limite.

A Tabela \ref{table:primeiro10} mostra o resultado desse experimento. Para cada família de fórmulas temos uma linha na tabela para as satisfatíveis, representada pelo sufixo \_n, e uma linha para as insatisfatíveis, representada pelo sufixo \_p. Cada coluna representa uma heurística de seleção de cláusula. Cada célula da tabela informa a quantidade de fórmulas que o provador proveu solução dentro do tempo limite e a média de tempo entre essas.

\begin{table*}[t]
	\[
\begin{array}{|c|c|c|c|c|c|}
	\hline - & \textit{mais antiga} & \textit{mais nova} & \textit{mínima} & \textit{máxima} & \textit{menor} \\

	\hline \textnormal{k\_branch\_n} & 2(59.6) & 1(0.06) & 1(0.00) & 1(0.05) & 2(4.34) \\
	\hline \textnormal{k\_branch\_p} & 2(1.52) & 2(88.84) & 2(29.37)& 2(101.32) & 3(1.4) \\
	\hline \textnormal{k\_d4\_n} & 9(14.5) & 7(44.42) & 8(27.48)& 8(21.68) & 9(13.5) \\
	\hline \textnormal{k\_d4\_p} & 21(0.16) & 21(0.08) & 21(0.03) & 21(0.57) & 21(0.08) \\
	\hline \textnormal{k\_dum\_n} & 21(0.03) & 21(0.05) & 21(0.04) & 21(0.12) & 21(0.08) \\
	\hline \textnormal{k\_dum\_p} & 21(0.09) & 21(0.14) & 21(0.51) & 21(0.34) & 21(0.06) \\
	\hline \textnormal{k\_grz\_n} & 21(9.56) & 17(3.01) & 17(5.9) & 17(16.35) & 21(3.15) \\
	\hline \textnormal{k\_grz\_p} & 21(0.02) & 21(0.01) & 21(0.01) & 21(0.01) & 21(0.01) \\
	\hline \textnormal{k\_lin\_n} & 21(0.01) & 21(0.01) & 21(0.01) & 21(0.01) & 21(0.02) \\
	\hline \textnormal{k\_lin\_p} & 21(0.01) & 21(0.01) & 21(0.01) & 21(0.01) & 21(0.03) \\
	\hline \textnormal{k\_path\_n} & 21(0.02) & 21(0.02) & 21(0.02) & 21(0.02) & 21(0.03) \\
	\hline \textnormal{k\_path\_p} & 21(0.03) & 21(0.02) & 21(0.02) & 21(0.02) & 21(0.02) \\
	\hline \textnormal{k\_ph\_n} & 3(4.08) & 3(10.88) & 3(1.85) & 3(5.81) & 3(0.65) \\
	\hline \textnormal{k\_ph\_p} & 2(0.02) & 2(0.00) & 3(14.82) & 3(89.8) & 3(0.03) \\
	\hline \textnormal{k\_poly\_n} & 16(50.45) & 8(27.89) & 14(42.53) & 13(39.82) & 16(52.46) \\
	\hline \textnormal{k\_poly\_p} & 15(36.83) & 10(38.17) & 14(45.65) & 14(39.16) & 15(43.32) \\
	\hline \textnormal{k\_t4p\_n} & 21(0.06) & 21(0.16) & 21(0.1) & 21(0.06) & 21(0.05) \\
	\hline \textnormal{k\_t4p\_p} & 21(0.04) & 21(0.1) & 21(0.03) & 21(0.04) & 21(0.02) \\
	
	\hline
\end{array}
	\]
\caption{Fórmulas resolvidas em até 300 seg e tempo médio em segundos.}
\label{table:primeiro10}
\end{table*}

Esses resultados indicam que o KSP é muito eficiente para a maioria das famílias independente da estratégia utilizada, mas muito pode ser melhorado em k\_branch, k\_d4, k\_ph e k\_poly.

\section{Trabalhos anteriores}
Uma análise de heurísticas de seleção de cláusula para provadores baseados em saturação, assim como KSP, foi feita por \cite{stephan}. O experimento foi feito no provador E sobre 13774 fórmulas do \textit{benchmark} TPTP \cite{TPTP} com 300 segundos de tempo limite.

As heurísticas avaliadas foram: \textit{Mais antiga}, Contagem de Símbolos e Ordenada. Em Contagem de Símbolos, é atribuído um peso a cada símbolo e é selecionada uma cláusula com a menor soma de pesos. No caso em que todos os símbolos têm peso um esta variação é idêntica a \textit{menor} usada no KSP. Ordenada é uma variação de Contagem de Símbolos onde é preferida cláusulas com o menor número de literais maximais.

Também foram analisadas várias intercalações de duas dessas heurísticas em distribuições diferentes. Por exemplo, a cada 11 seleções de cláusulas, 10 são feitas pela heurística Contagem de Símbolo e 1 é feita pela heurística \textit{Mais antiga}.

São apresentados vários resultados como número de fórmulas resolvidas, tamanho da prova, número de inferências na prova, etc para todas as estratégias utilizadas. Vemos que a maioria das fórmulas resolvidas foram resolvidas em poucos segundos mesmo com 300 segundos disponíveis. Os experimentos apontam não haver melhora em performance ao usar diferentes funções de peso para os literais. Todas as estratégias de contagem de símbolos tiveram ganho significativo de desempenho quando intercaladas com \textit{Mais antiga}. Selecionar sempre cláusulas dadas na entrada primeiro melhorou performance no geral, mas não tanto com estratégias utilizando \textit{Mais antiga}.

\section{Primeira proposta} \label{secao:1proposta}
Baseado no trabalho de \cite{stephan}, nossa primeira proposta de seleção de cláusula será intercalação de \textit{menor} e \textit{mais antiga}, já implementadas no KSP, numa razão $p$:$q$ informada pelo usuário. Dessa forma, as primeiras $p$ execuções da função {\sf given} serão conforme \textit{menor}, as próximas $q$ serão conforme \textit{mais antiga}, as próximas $p$ conforme \textit{menor} e assim por diante.

A implementação dessa proposta não é muito complexa por ser intercalação de métodos já implementados e nos permite comparar os resultados do KSP com os encontrados por \cite{stephan} no provador E.

%- Inferencia por nivel.
%- escrever
%- pensar num jeito de normalizar pra ver onde ta indo bem

Na tabela \ref{table:300stam5ant1}, apresentamos os resultados para razão de 5 seleções por \textit{Menor tamanho} para cada seleção por \textit{Mais antiga}. Usamos sat como abreviação de satisfatível e unsat como abreviação de unsatisfatível.
\begin{table*}[t]
	\[
\begin{array}{|c|c|c|c|c|}
	\hline - & \textnormal{Tempo médio(sat)} & \textnormal{Resolvidas(sat)} & \textnormal{Tempo médio(unsat)} & \textnormal{Resolvidas(unsat)} \\
	
	\hline \textnormal{k\_branch} & 5.23 & 2 & 3.68 & 3 \\
	\hline \textnormal{k\_d4} & 13.93 & 9 & 0.021 & 21 \\
	% \hline \textnormal{k\_dum} & 0.017 & 21 & 0.012 & 21 \\
	% \hline \textnormal{k\_grz} & 1.37 & 21 & 0.00 & 21 \\
	% \hline \textnormal{k\_lin} & 0.00 & 21 & 0.00 & 21 \\
	% \hline \textnormal{k\_path} & 0.01 & 21 & 0.01 & 21 \\
	\hline \textnormal{k\_ph} & 1.46 & 3 & 0.05 & 3 \\
	\hline \textnormal{k\_poly} & 60.24 & 16 & 38.16 & 15 \\
	% \hline \textnormal{k\_t4p} & 0.04 & 21 & 0.01 & 21 \\
	
	\hline
	
\end{array}
	\]
\caption{Fórmulas resolvidas com razão 5:1 entre tamanho e mais antiga em até 300 segundos e tempo médio em segundos.}
\label{table:300stam5ant1}
\end{table*} % using results from 20210103.163648_300 tam_5_ant_1

\section{Segunda proposta}
Como a seleção de cláusula em cada nível é feita de forma independente, podemos também usar algoritmos distintos em níveis distintos.

Com base nos resultados dos experimentos descritos na Seção~\ref{secao:1proposta}, propomos intercalação de \textit{menor} e \textit{mais antiga} com razão dinâmica para todo nível.

Cada nível terá um escalonador responsável por escolher uma razão eficiente. Como não conhecemos a melhor solução a priori, esta razão mudará ao longo da execução do provador. Neste experimento usamos somente o número de inferências feitas para determinar a troca de proporção na intercalação dos algoritmos.

Para o nível modal $ml$, o escalonador usará apenas a cardinalidade de $\Gamma^{lit}_{ml}$ para determinar qual razão aplicar. O usuário define $r_1, l_1, r_2, l_2, \ldots, r_x$, onde $r_i$ é uma razão descrita na Seção~\ref{secao:1proposta} e $l_i$ é uma sequência ordenada de limiares para troca de razões. Para cardinalidade menor ou igual a $l_1$ é usada razão $r_1$, para cardinalidade maior que $l_1$ e menor ou igual a $l_2$ é usada razão $r_2$ e assim por diante. Note que não há limite superior para a ultima razão, a $r_x$.

\begin{table*}[t]
	\[
\begin{array}{|c|c|c|c|c|}
	\hline - & \textnormal{Tempo médio(sat)} & \textnormal{Resolvidas(sat)} & \textnormal{Tempo médio(unsat)} & \textnormal{Resolvidas(unsat)} \\
	
	\hline \textnormal{k\_branch} & 5.27 & 2 & 11.64 & 3 \\
	\hline \textnormal{k\_d4} & 14.89 & 9 & 0.03 & 21 \\
	\hline \textnormal{k\_ph} & 2.04 & 3 & 0.02 & 3 \\
	\hline \textnormal{k\_poly} & 53.04 & 16 & 37.76 & 15 \\
	
	\hline
	
\end{array}
	\]
\caption{Fórmulas resolvidas em até 300 segundos e tempo médio em segundos.}
\label{table:300threshold1}
% using results from 20210316.061043_300
% 1, oldest; 20;
% 3, shortest, 2, oldest; 200;
% 5, shortest, 1, oldest; 10000;
% 10, shortest, 1, oldest; 30000;
% 1, oldest
\end{table*}

\section{Terceira proposta}
A regra de inferência GEN1 no laço principal do \ksp precisa de alguma cláusula com um literal que aparece num operador $\pos{\alpha}$, para qualquer $\alpha$. Para facilitar esta aplicação, vamos fazer seleção entre cláusulas com algum literal destes caso exista. Por exemplo, para o algoritmo de \testit{Menor tamanho}, vamos selecionar a cláusula de menor tamanho dentre as que têm algum literal de $\pos{\alpha}$.

%starexec

%dtodos os niveis usando msm razao 1:10
%razao diferente com base numa metrica
%selecao top com literal

%martin suda
