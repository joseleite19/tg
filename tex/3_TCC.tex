Neste capítulo propomos novos métodos de seleção de cláusula para o KSP.

\section{Trabalhos anteriores}
Uma análise de heurísticas de seleção de cláusula para provadores de saturação, assim como KSP, foi feita por \cite{stephan}. O experimento é feito sobre 13774 fórmulas do \textit{benchmark} TPTP \cite{TPTP} com 300 segundos de tempo limite.

As heurísticas avaliadas foram: \textit{First-in/First-out}, Contagem de Símbolos e Ordenada. Em \textit{First-in/First-out}, ou FIFO, é sempre selecionada cláusula não processada mais antiga. Em Contagem de Símbolos, é atribuído um peso a cada símbolo, por exemplo todos 1, e é selecionada uma cláusula com a menor soma de pesos. Ordenada é uma variação de Contagem de Símbolos onde é preferida cláusulas com o menor número de literais máximos.

Também foram analisadas várias intercalações de duas dessas heurísticas em distribuições diferentes. Por exemplo, a cada 11 seleções de cláusulas, 10 são feitas pela heurística Contagem de Símbolo e 1 é feita pela heurística FIFO.

São apresentados vários resultados como número de fórmulas resolvidas, tamanho da prova, número de inferencias na prova, etc para todas as estratégias utilizadas. Vemos que a maioria das fórmulas resolvidas foram resolvidas em poucos segundos mesmo com 300 segundos disponíveis. Os experimentos apontam nao haver melhora em performance ao usar diferentes funções de peso para os literais. Todas as estrategias de contagem de simbolo tiveram ganho significativo de desempenho quando intercaladas com FIFO. Selecionar sempre cláusulas iniciais primeiro melhorou performance no geral, mas não tanto com estratégias utilizando FIFO.

\section{Primeira proposta}
Contagem de símbolo + FIFO 10:1



\section{Primeira proposta}
Contagem de símbolo + FIFO 10:1
starexec

montar inferencia


todos os niveis usando msm razao 1:10
razao diferente com base numa metrica
selecao top com literal

martin suda
