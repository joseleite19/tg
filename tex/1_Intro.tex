% Escolha de clausulas para provador automatico de teorema baseado em resolucao
% descrever o problema - escolher clausula para resolucao
% area da computacao, definir prob de forma nao tecnica
% falar de ideias que ja existem e pq nao sao satisfatorias

%Lógica nos fornece ferramentas para criar e reconhecer argumentos válidos. Embora seja custoso % TODO citar
%formalizar problemas do mundo real para poder utilizar estas ferramentas, é desejável, principalmente em sistemas críticos, ter a certeza de que a solução aplicada está correta. Alguns exemplos onde isto é empregado são: verificação de \textit{hardware}, verificação de programas, verificação de protocolos etc. % TODO citar, e acho q ta estranho

%Formalmente, um argumento válido pode ser reescrito como prova de teorema.

%Cientistas da computação são apaixonados por automação, %TODO citar?
%então é natural que esforços para prova automática de teoremas sejam feitos. %TODO um pouco de historia e resolucao

Em computação, estudamos classes de problemas que ajudam a expressar quão difícil resolver um problema é, classificando por tempo ou espaço. P é a classe de problemas que podem ser resolvidos em tempo polinomial em uma máquina determinística. NP é a classe de problemas que podem ser resolvidos em tempo polinomial em uma máquina não deterministica. O problema de P vs NP consiste em determinar se P $\not=$ NP e tem implicações na matemática, criptografia. PSPACE é a classe de problemas que podem ser resolvidos em espaço polinomial. Dentre outras, as classes P e NP estão contidas dentro da classe PSPACE o que demonstra um pouco a sua representatividade. %TODO dar uma olhada no livro de automatos
Logica Modal proposicional $K_n$ pode expressar qualquer problema da classe PSPACE, ou seja, PSPACE-completo \cite{Ladner77}.

Representar problemas formalmente nos permite usar ferramentas para garantidamente formular ou reconhecer uma resposta.
%Lógica nos fornece ferramentas para garantidamente formular e reconhecer argumentos válidos.
A grande expressividade de $K_n$ permite minimização de Automatos Finitos Não determinísticos\cite{nfahard}, computar o Equilíbrio de Nash para qualquer jogo de dois jogadores em forma normal \cite{nashpspace}, ou até representar modelos de outras lógicas \cite{correspkn} o que possibilita seu uso desde engenharia de requisitos \cite{reqeng} a geração contos de fadas \cite{fairytale}.

Assim, é desejável ter ferramentar para resolver problemas em $K_n$ e que estas sejam eficientes. Para isso, muitos provadores de teoremas foram construídos para tal lógica. \ksp é um provador automático baseado em resolução descrito em \cite{Nalon2020}. Provadores automáticos são interessantes por diminuir o risco de erro humano na prova. Um dos passos de todo provador baseado em resolução é a seleção de cláusula. Muitas heurísticas para tal já foram estudadas mas ainda há muito a ser melhorado mesmo ao estado da arte \cite{stephan}.

%TODO "pq escolher o KSP?"

Neste trabalho queremos resolver eficientemente o problema de seleção de cláusula no \ksp.

%Area da computacao eh logica computacional.

No Capítulo \ref{2_Def} apresentamos toda a teoria necessária para o trabalho. O Capítulo \ref{3_TCC} descreve os experimentos feitos.

% breve descricao dos capitulos
% logica modal consegue expressar tudo em P-Space
% procurar aplicacoes
% dizer do q trata no artigo e a citacao
% prcurar ontologias, planejamento, verificacao de hardware, verificacao de protocolos, verificacao de software, especificacao de sistemas multi agente, verificacao de sistemas multiagente.
% importante ter ferramentas pra resolv prob
% apresentar oq ja existe do KSP
% apresentar oq pode ser melhorado

% ALC sem raciocinio local para individuos
