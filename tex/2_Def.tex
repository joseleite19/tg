	
Nesta seção apresentamos as definições básicas para o resto do texto.

\section{Linguagem}
Trabalharemos com a linguagem lógica modal K.

\begin{definition}
Seja $P = \{p, q, r, \dots\}$ um conjunto finito de símbolos proposicionais, $\Agents = \{1, 2, 3, \dots, n\}, n \in \mathbb{N}$. Definimos o conjunto de fórmulas $\mathcal{FBF}$ indutivamente.

\begin{itemize}
\item Se $\varphi \in \set{P}$ então $\varphi \in \set{FBF}$ e $\neg\varphi \in \set{FBF}$
\item Se $\varphi \in \set{FBF}$, $\psi \in \set{FBF}$ e $\agent \in \Agents$, então $(\varphi \land \psi) \in \set{FBF}$, $(\varphi \lor \psi) \in \set{FBF}$, $(\varphi \rightarrow \psi) \in \set{FBF}$, $\nec{\agent} \varphi \in \set{FBF}$, $\pos{\agent} \varphi \in \set{FBF}$ e $\neg\varphi \in \set{FBF}$.
\end{itemize}
\end{definition}


A semântica para lógica modal proposicional é dada por estruturas de Kripke. Uma estrutura de Kripke $M$ é da forma $M = (\set{W}, \set{R}_1, \ldots, \set{R}_{|\Agents|}, \pi)$, onde % TODO Cita
$\set{W}$ é um conjunto de mundos possíveis, $\pi:\set{W} \times \set{P} \rightarrow \{\ctrue, \cfalse\}$, $\set{R}_\agent \subseteq \set{W} \times \set{W}$ para todo $\agent \in \Agents$. Dizemos que uma fórmula $\varphi$ é satisfatível na lógica modal K sob um mundo $\st$ se, e somente se, $M,w \models \varphi$.

\begin{itemize}
\item $M,\st \models \varphi$, se e somente se $\varphi \in \set{P}$ e $\pi(\st, \varphi) = \ctrue$
\item $M,\st \models \neg\varphi$, se e somente se $M,\st \not\models \varphi$
\item $M,\st \models (\varphi \land \psi)$, se e somente se $M,\st \models \varphi$ e $M,\st \models \psi $
\item $M,\st \models (\varphi \lor \psi)$, se e somente se $M,\st \models \varphi$ ou $M,\st \models \psi $
\item $M,\st \models (\varphi \then \psi)$, se e somente se $M,\st \not\models \varphi$ ou $M,\st \models \psi $

\item $M,\st \models \pos{\agent}\varphi$, se e somente se $\exists\st', (\st, \st') \in \set{R}_\agent, M,\st' \models \varphi$
\item $M,\st \models \nec{\agent}\varphi$, se e somente se $\forall\st', (\st, \st') \in \set{R}_\agent, M,\st' \models \varphi$

\end{itemize}

% 1) definir literal e literal modal
% 2) definir polaridade

\begin{definition}
$\set{LP}$ é o conjunto de literais proposicionais e $\set{LM}$ é o conjunto de literais modais.
$\forall p \in \set{P}, \forall \agent \in \set{A}$, então $ p \in \set{LP}, \neg p \in \set{LP}, \nec{\agent} p \in \set{LM}, \nec{\agent} \neg p \in \set{LM}, \pos{\agent} p \in \set{LM}, \pos{\agent} \neg p \in \set{LM}$
\end{definition}

Seja $\Sigma = \{0, 1\}$, $\Sigma^*$ é o conjunto de todas as cadeias formadas com elementos de $\Sigma$. Em particular, $\epsilon$ representa a cadeia vazia.

\begin{definition}

$pol \colon \set{FBF} \times \set{FBF} \times \Sigma^* \mapsto \{\text{positiva, negativa}\}$. E seja $inv \colon \{\text{positiva, negativa}\} \mapsto \{\text{positiva, negativa}\}$ tal que $inv(positiva) = negativa$ e $inv(negativa) = positiva$. Para $\phi \in \set{FBF}, \varphi \in \set{FBF}, \psi \in \set{FBF}, \alpha \in \set{FBF}, s \in \Sigma^*, \agent \in \set{A}$.

\begin{itemize}
  \item[] $pol(\varphi, \varphi, \epsilon) = positiva$.
  \item[] Se $\phi = \psi \land \alpha$ ou $\phi = \psi \lor \alpha$, então $pol(\varphi, \psi, s0) = pol(\varphi, \alpha, s1) = pol(\varphi, \phi, s)$.
  \item[] Se $\phi = \psi \rightarrow \alpha$, então $pol(\varphi, \psi, s0) = inv(pol(\varphi, \phi, s))$ e $pol(\varphi, \alpha, s1) = pol(\varphi, \phi, s)$.
  \item[] Se $\phi = \pos{\agent}\psi$ ou $\phi = \nec{\agent}\psi$, então $pol(\varphi, \psi, s0) = pol(\varphi, \phi, s)$.
  \item[] Se $\phi = \neg\psi$, então $pol(\varphi, \psi, s0) = inv(pol(\varphi, \phi, s))$.
\end{itemize}

\end{definition}
A cadeia $s$ codifica a ocorrência de uma fórmula $\psi$ em uma fórmula $\varphi$.  Dizemos que a polaridade de $\psi$ em $\varphi$ na posição $s$ é $pol(\varphi, \psi, s)$.

\begin{definition}
O nível modal de uma fórmula. $mlevel \colon \set{FBF} \times \set{FBF} \times \Sigma^* \mapsto \mathbb{N}$

\begin{itemize}
  \item[] $mlevel(\varphi, \varphi, \epsilon) = 0$.
  \item[] Se $\phi = \psi \land \alpha$ ou $\phi = \psi \lor \alpha$ ou $\phi = \psi \rightarrow \alpha$, então $mlevel(\varphi, \psi, s0) = mlevel(\varphi, \alpha, s1) = mlevel(\varphi, \phi, s)$.
\item[] Se $\phi = \pos{\agent}\psi$ ou $\phi = \nec{\agent}\psi$, então $mlevel(\varphi, \psi, s0) = pol(\varphi, \phi, s)+1$.
  \item[] Se $\phi = \neg\psi$, então $mlevel(\varphi, \psi, s0) = mlevel(\varphi, \phi, s)$.
\end{itemize}
\end{definition}
Similar à definição de polaridade. Dizemos que o nível modal de $\psi$ em $\varphi$ na posição $s$ é $mlevel(\varphi, \psi, s)$.

% \section{Forma normal}
\section{Resolução}
% \subsection{Proposicional}

% \subsection{Modal}
% citar correcao e completude e terminacao do calculo
% implementacao do calculo
% loop principal do algoritmo
% como funciona
% quais estrategias eh baseado
% quais opcoes existentes que sao relevantes pro meu prob(escolha de clausulas e selecao de literais)

\section{Grafo}
Um grafo é um par um ordenado $(V, E)$, onde $E \subseteq V\times V$, chamamos $V$ o conjunto de vértices e $E$ o conjunto de arestas.

\section{Matroide}
Seja $X$ um conjunto de objetos e $I \subseteq 2^X$ o conjunto de conjuntos independentes tal que:
\begin{enumerate}
\item $\emptyset \in I$
\item $A \in I, B \subseteq A \implies B \in I$
\item Axioma do troco, $A \in I, B \in I, |B| > |A| \implies \exists x \in B \setminus A : A \cup \{x\} \in I$
\item Se $A \subseteq X$ e $I$ e $I'$ são conjuntos independentes maximais de A então $|I| = |I'|$
\end{enumerate}
Então $(X, I)$ é um matroide. O problema combinatório associado a ele é: Dada um função de peso $w(e) \geq 0 ~\forall e \in X$, encontre um subconjunto independente com maior soma de pesos possível.
