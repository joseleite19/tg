Nesta seção apresentamos as definições básicas para o resto do texto.

\section{Linguagem}
\subsection{Sintaxe}
Seja $P = \{p, q, r \dots\}$ um conjunto finito de símbolos proposicionais, $\Agents \subset \set{N}$ um conjunto finito, $\set{OB} = \{\land, \lor, \rightarrow, \leftrightarrow \}$ o conjunto de operadores binários e $\set{OU} = \{\nec{\agent}, \pos{\agent}, \neg \}$, onde $\agent \in \Agents$, o conjunto dos operadores unários. Definimos o conjunto de fórmulas $\mathcal{L}$ indutivamente.

\begin{itemize}
\item Se $\varphi \in \set{P}$ então $\varphi \in \set{L}$
\item Se $\varphi \in \set{L}$ e $* \in \set{OU}$, então $* \varphi \in \set{L}$
\item Se $\varphi \in \set{L}$, $\psi \in \set{L}$ e $* \in \set{OB}$, então $(\varphi * \psi) \in \set{L}$
\item $\top \in \set{L}$
\item $\bot \in \set{L}$
\end{itemize}

\subsection{Semântica}
A semântica para lógica modal proposicional é dada por estruturas de Kripke. Uma estrutura de Kripke $M$ é da forma $M = (\set{W}, \set{R}_1, \ldots, \set{R}_{|\Agents|}, \pi)$, onde % TODO Cita
$\set{W}$ é um conjunto de mundos possíveis, $\pi:\set{W} \times \set{P} \rightarrow \{\ctrue, \cfalse\}$, $\set{R}_\agent \subseteq \set{W} \times \set{W}$ para todo $\agent \in \Agents$. Uma fórmula $\varphi$ é satisfatível na lógica modal K sob um mundo $\st$ se, e somente se, $M,w \models \varphi$.

\begin{itemize}
\item $M,\st \models \top$
\item $M,\st \not\models \bot$
\item $M,\st \models \varphi$, se e somente se $\varphi \in \set{P}$ e $\pi(\st, \varphi) = \ctrue$
\item $M,\st \models \neg\varphi$, se e somente se $M,\st \not\models \varphi$
\item $M,\st \models (\varphi \land \psi)$, se e somente se $M,\st \models \varphi$ e $M,\st \models \psi $
\item $M,\st \models (\varphi \lor \psi)$, se e somente se $M,\st \models \varphi$ ou $M,\st \models \psi $
\item $M,\st \models (\varphi \then \psi)$, se e somente se $M,\st \not\models \varphi$ ou $M,\st \models \psi $
\item $M,\st \models (\varphi \ifonlyif \psi)$, se e somente se $M,\st \models (\varphi \then \psi)$ e $M,\st \models (\psi \then \varphi)$

\item $M,\st \models \pos{\agent}\varphi$, se e somente se $\exists\st', (\st, \st') \in \set{R}_\agent, M,\st' \models \varphi$
\item $M,\st \models \nec{\agent}\varphi$, se e somente se $\forall\st', (\st, \st') \in \set{R}_\agent, M,\st' \models \varphi$

\end{itemize}

\section{Forma normal}
\section{Resolução}
% citar correcao e completude e terminacao do calculo
% implementacao do calculo
% loop principal do algoritmo
% como funciona
% quais estrategias eh baseado
% quais opcoes existentes que sao relevantes pro meu prob(escolha de clausulas e selecao de literais)

\section{Matroide}
Seja $X$ um conjunto de objetos e $I \subseteq 2^X$ o conjunto de conjuntos independentes tal que:
\begin{enumerate}
\item $\emptyset \in I$
\item $A \in I, B \subseteq A \implies B \in I$
\item Axioma do troco, $A \in I, B \in I, |B| > |A| \implies \exists x \in B \setminus A : A \cup \{x\} \in I$
\item Se $A \subseteq X$ e $I$ e $I'$ são conjuntos independentes maximais de A então $|I| = |I'|$
\end{enumerate}
Então $(X, I)$ é um matroide. O problema combinatório associado a ele é: Dada um função de peso $w(e) \geq 0 ~\forall e \in X$, encontre um subconjunto independente com maior soma de pesos possível.
