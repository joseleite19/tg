
Neste capítulo apresentamos as definições básicas para o resto do texto.

\section{Linguagem}
Trabalharemos com a linguagem lógica modal $K_n$. Nesta seção descreveremos primeiro a sintaxe da linguagem, isto é, como podemos construir e reconhecer fórmula nesta linguagem. Depois descrevemos a sua semântica, isto é, como atribuímos significado à fórmulas. %TODO write a small introduction for the section and describe its structure, probably define syntax and semantics

\begin{definition}
	Seja $P = \{p, q, r, \dots\}$ um conjunto enumerável de símbolos proposicionais, $\Agents = \{1, 2, 3, \dots, n\}, n \in \mathbb{N}$. Definimos o conjunto de fórmulas bem formadas $\mathcal{FBF}$ indutivamente.
	
	\begin{itemize}
		\item Se $\varphi \in \set{P}$ então $\varphi \in \set{FBF}$
		\item Se $\varphi \in \set{FBF}$, $\psi \in \set{FBF}$ e $\agent \in \Agents$, então $(\varphi \land \psi) \in \set{FBF}$, $(\varphi \lor \psi) \in \set{FBF}$, $(\varphi \rightarrow \psi) \in \set{FBF}$, $\nec{\agent} \varphi \in \set{FBF}$, $\pos{\agent} \varphi \in \set{FBF}$ e $\neg\varphi \in \set{FBF}$
	\end{itemize}
\end{definition}

\begin{definition}
	Denotamos por $\set{LP}$ o conjunto de literais proposicionais e por $\set{LM}$ o conjunto de literais modais.
	$\forall p \in \set{P}, \forall \agent \in \set{A}$, temos que $ p \in \set{LP}, \neg p \in \set{LP}, \nec{\agent} p \in \set{LM}, \nec{\agent} \neg p \in \set{LM}, \pos{\agent} p \in \set{LM}, \pos{\agent} \neg p \in \set{LM}$
\end{definition}

\begin{definition}
	Uma cláusula proposicional é uma disjunção de literais proposicionais, ou seja, tem a forma $p_1 \lor p_2 \lor \cdots p_m$ com $p_i \in \set{P}$ e $m \in \mathbf{N}$. Caso $m = 0$ a cláusula é dita vazia. %TODO describe disjuntion, maybe create a lexical subsection before syntax
\end{definition}

Seja $\Sigma = \{0, 1\}$, $\Sigma^*$ é o conjunto de todas as cadeias formadas com elementos de $\Sigma$. Em particular, $\epsilon$ representa a cadeia vazia. Construiremos cadeias em $\Sigma^*$ para codificar a posição de ocorrência de uma subfórmula em uma fórmula. Seja $inv \colon \{\text{positiva, negativa}\} \mapsto \{\text{positiva, negativa}\}$ tal que $inv(\text{positiva}) = \text{negativa}$ e $inv(\text{negativa}) = \text{positiva}$.

\begin{definition}
	Definimos a polaridade de uma subfórmula pela função $pol \colon \set{FBF} \times \set{FBF} \times \Sigma^* \mapsto \{\text{positiva, negativa}\}$.  Para $\varphi, \chi_1, \chi_2 \in \set{FBF}, s \in \Sigma^*, \agent \in \set{A}, val \in \{\text{positiva, negativa}\}$.
	
	\begin{itemize}
		\item $pol(\varphi, \varphi, \epsilon) = \text{positiva}$.
		\item Se $pol(\varphi, \chi_1 \lor \chi_2, s) = val$, então $pol(\varphi, \chi_1, s0) = pol(\varphi, \chi_2, s1) = val$
		\item Se $pol(\varphi, \chi_1 \land \chi_2, s) = val$, então $pol(\varphi, \chi_1, s0) = pol(\varphi, \chi_2, s1) = val$
		\item Se $pol(\varphi, \chi_1 \rightarrow \chi_2, s) = val$, então $pol(\varphi, \chi_1, s0) = inv(val)$ e $pol(\varphi, \chi_2, s1) = val$
		\item Se $pol(\varphi, \pos{\agent}\chi_1, s) = val$, então $pol(\varphi, \chi_1, s0) = val$
		\item Se $pol(\varphi, \nec{\agent}\chi_1, s) = val$, então $pol(\varphi, \chi_1, s0) = val$
		\item Se $pol(\varphi, \neg\chi_1, s) = val$, então $pol(\varphi, \chi_1, s0) = inv(val)$
	\end{itemize}
	
\end{definition}
Dizemos que a polaridade de $\psi$ em $\varphi$ na posição $s$ é $pol(\varphi, \psi, s)$.

\begin{definition}
	Definimos o nível modal de uma subfórmula pela função $mlevel \colon \set{FBF} \times \set{FBF} \times \Sigma^* \mapsto \mathbb{N}$. Para $\varphi, \chi_1, \chi_2 \in \set{FBF}, s \in \Sigma^*, \agent \in \set{A}, val \in \mathbb{N}$.
	
	\begin{itemize}
		\item $mlevel(\varphi, \varphi, \epsilon) = 0$.
		\item Se $mlevel(\varphi, \chi_1 \lor \chi_2, s) = val$ ou $mlevel(\varphi, \chi_1 \land \chi_2, s) = val$ ou $mlevel(\varphi, \chi_1 \rightarrow \chi_2, s) = val$, então $mlevel(\varphi, \chi_1, s0) = mlevel(\varphi, \chi_2, s1) = val$
		\item Se $mlevel(\varphi, \pos{\agent}\chi_1, s) = val$ ou $mlevel(\varphi, \nec{\agent}\chi_1, s) = val$, então $mlevel(\varphi, \chi_1, s0) = val+1$
		\item Se $mlevel(\varphi, \neg\chi_1, s) = val$, então $mlevel(\varphi, \chi_1, s0) = val$
	\end{itemize}
\end{definition}


Dizemos que o nível modal de $\chi_1$ em $\varphi$ na posição $s$ é $mlevel(\varphi, \psi, s)$.

A semântica para lógica modal proposicional é dada por estruturas de Kripke. Uma estrutura de Kripke $M$ é da forma $M = (\set{W}, \st_0, \set{R}_1, \ldots, \set{R}_{|\Agents|}, \pi)$, onde % TODO Cita
$\set{W}$ é um conjunto%TODO colocar mundo? (mundos possíveis)
, $\st_0 \in \set{W}$, $\pi:\set{W} \times \set{P} \rightarrow \{\vtrue, \vfalse\}$, $\set{R}_\agent \subseteq \set{W} \times \set{W}$ para todo $\agent \in \Agents$. Dizemos que uma fórmula $\varphi$ é satisfeita na lógica modal $K_n$ no modelo $M$ no mundo $\st$ se, e somente se, $\langle M,w \rangle \models \varphi$, conforme segue:

\begin{itemize}
	\item $\langle M,\st \rangle \models p$ se, e somente se $\pi(\st, p) = \vtrue$, para $p \in \set{P}$
	\item $\langle M,\st \rangle \models \neg\varphi$ se, e somente se $\langle M,\st \rangle \not\models \varphi$
	\item $\langle M,\st \rangle \models (\varphi \land \psi)$ se, e somente se $\langle M,\st \rangle \models \varphi$ e $\langle M,\st \rangle \models \psi$
	\item $\langle M,\st \rangle \models (\varphi \lor \psi)$ se, e somente se $\langle M,\st \rangle \models \varphi$ ou $\langle M,\st \rangle \models \psi$
	\item $\langle M,\st \rangle \models (\varphi \then \psi)$ se, e somente se $\langle M,\st \rangle \not\models \varphi$ ou $\langle M,\st \rangle \models \psi$
	
	\item $\langle M,\st \rangle \models \pos{\agent}\varphi$ se, e somente se $\exists\st', (\st, \st') \in \set{R}_\agent, \langle M,\st' \rangle \models \varphi$
	\item $\langle M,\st \rangle \models \nec{\agent}\varphi$ se, e somente se $\forall\st', (\st, \st') \in \set{R}_\agent, \langle M,\st' \rangle \models \varphi$
	
\end{itemize}

Uma fórmula $\varphi$ é localmente satisfatível se existe um modelo $M$ tal que $\langle M,\st_0 \rangle \models \varphi$. Uma formula $\varphi$ é globalmente satisfatível se existe um modelo $M$ tal que para todo $\st \in \set{W}$ temos que $\langle M,\st \rangle \models \varphi$. Escrevemos $M \models \varphi$ se, e somente se, $\langle M, \st_0 \rangle \models \varphi$.

Na Figura~\ref{table:exemplomodelo1} apresentamos a visualização de um modelos possível para a fórmula $(p \land \nec{1} (p \land \pos{2}q))$. Na Figura~\ref{table:exemplomodelo2}, um modelo para $\nec{1}\nec{2}\pos{1}\pos{1}\nec{3}(p \then q) \lor \pos{1}\nec{3}(p \then q) \land \nec{1}\nec{1}(\pos{1}\neg q \lor \nec{1}q$.

\begin{figure*}
\[
\begin{tabular}{cc}
\begin{tikzpicture}
\begin{scope}[every node/.style={circle,thick,draw}, node distance=2cm]
    \node (w0)  {$w_0$};
    \node (w1) [right of=w0,yshift=1cm] {$w_1$};
    \node (w2) [right of=w0,yshift=-1cm] {$w_2$};
\end{scope}

\begin{scope}[>={Stealth[black]},
el/.style = {inner sep=2pt, align=left, sloped},
every label/.append style = {font=\tiny},
              every node/.style={fill=white,circle},
              every edge/.style={draw=black,very thick}]
    \path [->] (w0) edge node[el,above] {$1$} (w1);
    \path [->] (w0) edge node[el,below] {$1$} (w2);
    \path [->] (w1) edge node[el,above] {$2$} (w2);
    \path [->] (w2) edge [loop right] node[el,above] {$2$} (w2);    
\end{scope}
\end{tikzpicture}
&
\begin{tabular}{|c|c|c|c|c|c|}
	\hline $\pi$ & $w_0$ & $w_1$ & $w_2$ \\
	\hline \textnormal{p} & $\vfalse$ & $\vtrue$ & $\vfalse$ \\
	\hline \textnormal{q} & $\vfalse$ & $\vtrue$ & $\vtrue$ \\
        \hline
\end{tabular}
\\
\end{tabular}
\]
\caption{Primeiro exemplo de modelo.}
\label{table:exemplomodelo1}
\end{figure*}

\begin{figure*}
\[
\begin{tabular}{cc}
\begin{tikzpicture}
\begin{scope}[every node/.style={circle,thick,draw}]
    \node (w0) {$w_0$};
    \node (w1) [right of=w0, xshift=2cm] {$w_1$};
    \node (w2) [below of=w0,yshift=-2cm] {$w_2$};
    \node (w3) [below of=w1,yshift=-2cm]{$w_3$};
    \node (w4) [right of=w1, xshift=2cm] {$w_4$};
\end{scope}

\begin{scope}[>={Stealth[black]},
el/.style = {inner sep=2pt, align=left, sloped},
every label/.append style = {font=\tiny},
              every node/.style={fill=white,circle},
              every edge/.style={draw=black,very thick}]
    \path [->] (w0) edge node[el,above] {$1$} (w1);
    \path [->] (w0) edge [bend right] node[el,below] {$1$} (w2);
    \path [->] (w1) edge node[el,above] {$1$} (w4);
    \path [->] (w1) edge node[el,below] {$2$} (w2);
    \path [->] (w1) edge node[el,above] {$3$} (w3);
    \path [->] (w2) edge node[el,below] {$1$} (w0);
    \path [->] (w2) edge [loop right] node {$2$} (w2);    
\end{scope}
\end{tikzpicture}
&
\begin{tabular}{|c|c|c|c|c|c|}
	\hline $\pi$ & $w_0$ & $w_1$ & $w_2$ & $w_3$ & $w_4$ \\

	\hline \textnormal{p} & $\vfalse$ & $\vfalse$ & $\vtrue$ & $\vtrue$ & $\vfalse$ \\
	\hline \textnormal{q} & $\vfalse$ & $\vfalse$ & $\vfalse$ & $\vfalse$ & $\vfalse$ \\

	\hline
\end{tabular}
\\
\end{tabular}
\]
\caption{Segundo exemplo de modelo.}
\label{table:exemplomodelo2}
\end{figure*}


\begin{definition}
	Para um modelo $M$, definimos a profundidade de um mundo pela função $depth \colon \set{W} \mapsto \mathbb{N}$, onde $depth(w)$ é tamanho do menor caminho de $w_0$ a $w$ no grafo $(\set{W}, \bigcup_{i=1}^n \set{R}_i)$. Chamamos de nível modal a classe de equivalência com todos os mundos com mesma profundidade.
\end{definition}

Podemos reduzir o problema de satisfatibilidade global ao problema de satisfatibilidade local com a extensão da linguagem $K_n$ pelo operador universal $\nec{*}$. A semântica deste operador é definida como $\langle M,\st \rangle \models \nec{*}\varphi$ se, e somente se, para todo $\st' \in \set{W}$, $\langle M,\st' \rangle \models \varphi$, onde $M = (\set{W}, \st_0, \set{R}_1, \ldots, \set{R}_{|\Agents|}, \pi)$.

\begin{definition}
	Uma fórmula está na forma normal negada caso seja formada somente por símbolos proposicionais, $\neg$, $\land$, $\lor$, $\nec{\agent}$ e $\pos{\agent}$ para $\agent \in \set{A}$, e a negação só é aplicada a símbolos proposicionais.
\end{definition}

É importante ressaltar se $\varphi \in \set{FBF}$ não está na forma normal negada, então $\varphi$ pode ser reescrita como $\psi \in \set{FBF}$ na forma normal negada com semântica equivalente. Isto é, para todo $\langle M, w \rangle$, $\langle M, w \rangle \models \varphi$ se, e somente se, $\langle M, w \rangle \models \psi$.

\section{Forma Normal Separada em Níveis Modais}
O cálculo a ser apresentado utiliza uma outra linguagem chamada de Forma Normal Separada em Níveis Modais ($SNF_{ml}$).

\begin{definition}
	Uma fórmula em $SNF_{ml}$ é uma conjunção de cláusulas. Para $ml \in \Nat \cup \{*\}$ e $l_1, l_2 \in \set{LP}$, cada cláusula está em um dos três formatos:
	\begin{itemize}
		\item $ml: c$, onde $c$ é uma cláusula proposicional
		\item $ml: l_1 \rightarrow \nec{a}l_2$
		\item $ml: l_1 \rightarrow \pos{a}l_2$
	\end{itemize}
\end{definition}

A satisfatibilidade de uma fórmula em $SNF_{ml}$ é definida a partir da satisfatibilidade em $K_n$. Sejam $ml: \varphi$ e $ml: \psi$ cláusulas em $SNF_{ml}$ e $M$ um modelo na lógica $K_n$.

\begin{itemize}
	\item $M \models *: \varphi$ se, e somente se, $M \models \nec{*}\varphi$.
	\item $M \models (ml: \varphi) \land (ml: \psi)$ se, e somente se, $M \models ml: \varphi$ e $M \models ml: \psi$.
	\item $M \models ml: \varphi$ se, e somente se, para todo $\st$ tal que $depth(\st) = ml$, temos $\langle M, \st \rangle \models ml: \varphi$.
\end{itemize}

Uma função de tradução de $K_n$ para $SNF_{ml}$ bem como a prova de que a tradução de uma fórmula preserva satisfatibilidade podem ser encontradas em \cite{nalontocl}.

Seja $\ge$ uma ordem total sobre os símbolos proposicionais. Estendemos esta ordem para os literais da seguinte forma: Se $p \in \set{P}$, então $\neg p \ge p$; Se $p, q \in \set{P}, p \not= q$ e $p \ge \neg q$, então $p \ge q$. %TODO check this. "p >= !q porque >= é transitiva(ou seja, você obtemo que está definido a partir do fato de que !p >= q e p >= q)

\begin{definition}
	O literal $l$ é máximo em $\varphi \in SNF_{ml}$ se e somente se $l$ ocorre em $\varphi$ e não há $l_2 \not= l$ em $\varphi$ tal que $l_2 \ge l$.
\end{definition}

Note que podemos escolher qualquer ordem sobre os símbolos proposicionais, assim um literal $l$ pode ser máximo numa ordem e não ser máximo em outra ordem.

\begin{definition}
	O tamanho de uma cláusula em $SNF_{ml}$ é a cardinalidade do conjunto contendo somente os literais na cláusula.
\end{definition}

\section{Regras de inferência} \label{secao:regrasinf}
O cálculo dedutivo baseado em resolução $RES_{ml}$ para lógica $K_n$ foi descrito em \cite{nalontocl}. Para simplificar a descrição das regras de inferência faremos uso de uma função parcial de unificação $\sigma \colon P(\Nat \cup \{*\}) \mapsto \Nat \cup \{*\}$, tal que $\sigma(\{ml, *\}) = ml$, $\sigma(\{ml\}) = ml$, e indefinida caso contrário. As regras de inferência de $RES_{ml}$ são apresentadas na Figura~\ref{figura:regrasinf}, onde $*-1=*$ e $m$ pode ser 0. Essas regras só podem ser aplicadas se o resultado da unificação for definido. Demostrações de correção e corretude do cálculo podem ser encotradas em \cite{nalontocl}.

\begin{figure*}[t]
	\[
	\begin{array}{c}
	\begin{array}{lll}
	\textnormal{[LRES]} & \textnormal{[MRES]} & \textnormal{[GEN2]} \\
	\begin{array}{c}
	ml: (D \lor l) \\
	ml': (D' \lor \neg l) \\
	\hline
	\sigma(\{ml, ml'\}): D \lor D' \\
	\end{array}
	&
	
	\begin{array}{c}
	ml: (l_1 \rightarrow \nec{\agent} l) \\
	ml': (l_2 \rightarrow \pos{\agent} \neg l) \\
	\hline
	\sigma(\{ml, ml'\}): \neg l_1 \lor \neg l_2 \\
	\end{array}
	&
	
	\begin{array}{c}
	ml_1: (l_1' \rightarrow \nec{\agent} l_1) \\
	ml_2: (l_2' \rightarrow \nec{\agent} \neg l_1) \\
	ml_3: (l_3' \rightarrow \pos{\agent} l_2) \\
	\hline
	\sigma(\{ml_1, ml_2, ml_3\}): \neg l_1' \lor \neg l_2' \lor \neg l_3' \\
	\end{array}
	
	
	\\ \end{array} \\ \\
	
	\begin{array}{ll}
	\textnormal{[GEN1]} & \textnormal{[GEN3]} \\
	
	\begin{array}{c}
	ml_1: (l_1' \rightarrow \nec{\agent} l_1) \\
	\vdots \\
	ml_m: (l_m' \rightarrow \nec{\agent} \neg l_m) \\
	ml_{m+1}: (l' \rightarrow \pos{\agent} \neg l) \\
	ml_{m+2}: (l_1 \lor \ldots \lor l_m \lor l) \\
	\hline
	ml: \neg l_1' \lor \ldots \lor \neg l_m' \lor \neg l' \\
	ml = \sigma(\{ml_1, \ldots, ml_{m+1}, ml_{m+2}-1\}) \\
	\end{array}
	&
	\begin{array}{c}
	ml_1: (l_1' \rightarrow \nec{\agent} l_1) \\
	\vdots \\
	ml_m: (l_m' \rightarrow \nec{\agent} \neg l_m) \\
	ml_{m+1}: (l' \rightarrow \pos{\agent} l) \\
	ml_{m+2}: (l_1 \lor \ldots \lor l_m) \\
	\hline
	ml: \neg l_1' \lor \ldots \lor \neg l_m' \lor \neg l' \\
	ml = \sigma(\{ml_1, \ldots, ml_{m+1}, ml_{m+2}-1\}) \\
	\end{array}
	
	\\ \end{array} \\
	\end{array} 
	\]
	\caption{Regras de inferência}
	\label{figura:regrasinf}
\end{figure*}



\section{Algoritmo}
Neste trabalho usaremos o KSP \cite{Nalon2020}, um provador baseado no cálculo visto na Seção ~\ref{secao:regrasinf} e que determina a satisfatibilidade de fórmulas em $K_n$. Caso insatisfatível%TODO define unsatisfiability
, uma prova é fornecida. Embora o KSP tome fórmulas em $K_n$ como entrada, este faz a tradução para $SNF_{ml}$ e todo o cálculo é feito nessa linguagem.

KSP utiliza uma variação de conjunto de suporte%TODO add reference
, técnica que restringe os candidatos possíveis para resolução. A demostração da completude dessa restrição para $SNF_{ml}$ pode ser encontrada em \cite{nalontocl}. %TODO check proof

Na versão para lógica clássica de conjunto de suporte, o conjunto de cláusulas $\Delta$ é particionado em dois conjuntos $\Gamma$, o conjunto de suporte ou não processado, e $\Lambda$, o conjunto \textit{usable} ou processado. Cláusulas são selecionadas de $\Gamma$ e resolvidas com cláusulas em $\Lambda$. A cláusula selecionada é removida de $\Gamma$ e acrescentada a $\Lambda$. Os resolventes produzidos são inseridos em $\Gamma$.

Na extensão para $SNF_{ml}$, onde as clásulas são rotuladas pelo nível modal, as clásulas de todo nível modal $ml$ são particionadas em três conjuntos $\Gamma^{lit}_{ml}$, $\Lambda^{lit}_{ml}$ e $\Lambda^{mod}_{ml}$. Cláusulas proposicionais são particionadas em $\Gamma^{lit}_{ml}$ e $\Lambda^{lit}_{ml}$ como no caso em lógica clássica. Cláusulas modais são armazenadas em $\Lambda^{mod}_{ml}$. Note que nenhuma regra de inferência descrita na Seção \ref{secao:regrasinf} produz novas cláusulas modais.

A seguir, descrevemos o algoritmo implementado no KSP.

\begin{algorithm}[H]
	\SetAlgoLined
	\Entrada{Fórmula em $K_n$}
	\Saida{Satisfatibilidade da fórmula}
	preprocessamento-da-entrada\;
	tradução-para-SNF\;
	preprocessamento-de-clausulas\;

	$\Gamma^{lit} \leftarrow \bigcup \Gamma^{lit}_{ml}$\;
	\Enqto{$\Gamma^{lit} \not= \emptyset$}{
		\ParaTodo{nível modal $ml$}{
			$clausula$ $\leftarrow$ given($ml$)\;
			\Se{não \textrm{redundante}($clausula$)}{
				GEN1($clausula, ml, ml - 1$)\;
				GEN3($clausula, ml, ml - 1$)\;
				LRES($clausula, ml, ml$)\;
				$\Lambda^{lit}_{ml} \leftarrow \Lambda^{lit}_{ml} \cup \{clausula\}$\;
			}
			$\Gamma^{lit}_{ml} \leftarrow \Gamma^{lit}_{ml} \setminus \{clausula\}$\;
			\Se{$0: \cfalse \in \Gamma^{lit}_{0}$}{%TODO undefined
				\Retorna insatisfatível\;
			}
			$\Gamma^{lit} \leftarrow \bigcup \Gamma^{lit}_{ml}$\;
		}
		\Retorna satisfatível\;
	}
	\caption{KSP-Proof-Search}
\end{algorithm}

As Linhas 1-3 aplicam algumas regras de simplificação, traduzem a fórmula para linguagem $SNF_{ml}$ e constroem os conjuntos \textit{usable} e de suporte. As Linhas 9-11 aplicam regras de inferências descritas na Seção~\ref{secao:regrasinf}.

A função {\sf given}, Linha 7, é responsável por escolher uma cláusula dentre todas as candidatas possíveis do conjunto de suporte. Cada nível modal é independente. Naturalmente, a função {\sf given} só considera as cláusulas do nível modal pedido. KSP implementa cinco variações dessa função: \textit{menor}, \textit{mais antiga}, \textit{mais nova}, \textit{mínima} e \textit{máxima}; e o usuário pode escolher qual deseja utilizar.

Na variação \textit{menor}, é selecionada uma cláusula com o menor tamanho de $\Gamma^{lit}_{ml}$.

Na variação \textit{mais antiga}, o provador armazenada a ordem nas quais as cláusulas foram adicionadas ao conjunto de cláusulas e é selecionada a que foi adicionada antes de todas em $\Gamma^{lit}_{ml}$.

\textit{Mais nova} é análoga a \textit{mais antiga}, mas é selecionada a que foi adicionada depois de todas as outras.

Em \textit{mínima}, é escolhida uma cláusula com o menor tamanho dentre as cláusulas com o menor literal máximo em $\Gamma^{lit}_{ml}$.

Em \textit{máxima}, é feita escolha análoga a \textit{mínima} mas dentre cláusulas com o maior literal máximo em $\Gamma^{lit}_{ml}$.

Neste trabalho propomos novos métodos para seleção de cláusulas e comparamos com os métodos previamente utilizados no KSP. Para comparação usaremos o LWB \cite{lwb}, que descreve geradores de \textit{benchmark} de tamanho arbitrário para nove famílias de fórmulas.


%\section{Grafo}
%Um grafo é um par um ordenado $(V, E)$, onde $E \subseteq V\times V$, chamamos $V$ o conjunto de vértices e $E$ o conjunto de arestas.

%\section{Matroide}
%Seja $X$ um conjunto de objetos e $I \subseteq 2^X$ o conjunto de conjuntos independentes tal que:
%\begin{enumerate}
%	\item $\emptyset \in I$
%	\item $A \in I, B \subseteq A \implies B \in I$
%	\item Axioma do troco, $A \in I, B \in I, |B| > |A| \implies \exists x \in B \setminus A : A \cup \{x\} \in I$
%	\item Se $A \subseteq X$ e $I$ e $I'$ são conjuntos independentes maximais de A então $|I| = |I'|$
%\end{enumerate}
%Então $(X, I)$ é um matroide. O problema combinatório associado a ele é: Dada um função de peso $w(e) \geq 0 ~\forall e \in X$, encontre um subconjunto independente com maior soma de pesos possível.
