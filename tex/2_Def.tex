
Nesta seção apresentamos as definições básicas para o resto do texto.

\section{Linguagem}
Trabalharemos com a linguagem lógica modal K.

\begin{definition}
	Seja $P = \{p, q, r, \dots\}$ um conjunto finito de símbolos proposicionais, $\Agents = \{1, 2, 3, \dots, n\}, n \in \mathbb{N}$. Definimos o conjunto de fórmulas $\mathcal{FBF}$ indutivamente.
	
	\begin{itemize}
		\item Se $\varphi \in \set{P}$ então $\varphi \in \set{FBF}$ e $\neg\varphi \in \set{FBF}$
		\item Se $\varphi \in \set{FBF}$, $\psi \in \set{FBF}$ e $\agent \in \Agents$, então $(\varphi \land \psi) \in \set{FBF}$, $(\varphi \lor \psi) \in \set{FBF}$, $(\varphi \rightarrow \psi) \in \set{FBF}$, $\nec{\agent} \varphi \in \set{FBF}$, $\pos{\agent} \varphi \in \set{FBF}$ e $\neg\varphi \in \set{FBF}$.
	\end{itemize}
\end{definition}

\begin{definition}
	$\set{LP}$ é o conjunto de literais proposicionais e $\set{LM}$ é o conjunto de literais modais.
	$\forall p \in \set{P}, \forall \agent \in \set{A}$, então $ p \in \set{LP}, \neg p \in \set{LP}, \nec{\agent} p \in \set{LM}, \nec{\agent} \neg p \in \set{LM}, \pos{\agent} p \in \set{LM}, \pos{\agent} \neg p \in \set{LM}$
\end{definition}

\begin{definition}
	Uma cláusula proposicional é uma disjunção de literais proposicionais.
\end{definition}

Seja $\Sigma = \{0, 1\}$, $\Sigma^*$ é o conjunto de todas as cadeias formadas com elementos de $\Sigma$. Em particular, $\epsilon$ representa a cadeia vazia. Construiremos cadeias em $\Sigma^*$ para codificar a posição de ocorrência de uma subfórmula em uma fórmula. Seja $inv \colon \{\text{positiva, negativa}\} \mapsto \{\text{positiva, negativa}\}$ tal que $inv(\text{positiva}) = \text{negativa}$ e $inv(\text{negativa}) = \text{positiva}$.

\begin{definition}
	Definimos a polaridade de uma subfórmula pela função $pol \colon \set{FBF} \times \set{FBF} \times \Sigma^* \mapsto \{\text{positiva, negativa}\}$.  Para $\varphi, \chi_1, \chi_2 \in \set{FBF}, s \in \Sigma^*, \agent \in \set{A}, val \in \{\text{positiva, negativa}\}$.
	
	\begin{itemize}
		\item $pol(\varphi, \varphi, \epsilon) = \text{positiva}$.
		\item Se $pol(\varphi, \chi_1 \lor \chi_2, s) = val$, então $pol(\varphi, \chi_1, s0) = pol(\varphi, \chi_2, s1) = val$
		\item Se $pol(\varphi, \chi_1 \land \chi_2, s) = val$, então $pol(\varphi, \chi_1, s0) = pol(\varphi, \chi_2, s1) = val$
		\item Se $pol(\varphi, \chi_1 \rightarrow \chi_2, s) = val$, então $pol(\varphi, \chi_1, s0) = inv(val)$ e $pol(\varphi, \chi_2, s1) = val$
		\item Se $pol(\varphi, \pos{\agent}\chi_1, s) = val$, então $pol(\varphi, \chi_1, s0) = val$
		\item Se $pol(\varphi, \nec{\agent}\chi_1, s) = val$, então $pol(\varphi, \chi_1, s0) = val$
		\item Se $pol(\varphi, \neg\chi_1, s) = val$, então $pol(\varphi, \chi_1, s0) = inv(val)$
	\end{itemize}
	
\end{definition}
Dizemos que a polaridade de $\chi_1$ em $\varphi$ na posição $s$ é $pol(\varphi, \chi_1, s)$.

\begin{definition}
	Definimos o nível modal de uma subfórmula pela função $mlevel \colon \set{FBF} \times \set{FBF} \times \Sigma^* \mapsto \mathbb{N}$. Para $\varphi, \chi_1, \chi_2 \in \set{FBF}, s \in \Sigma^*, \agent \in \set{A}, val \in \mathbb{N}$.
	
	\begin{itemize}
		\item $mlevel(\varphi, \varphi, \epsilon) = 0$.
		\item Se $mlevel(\varphi, \chi_1 \lor \chi_2, s) = val$ ou $mlevel(\varphi, \chi_1 \land \chi_2, s) = val$ ou $mlevel(\varphi, \chi_1 \rightarrow \chi_2, s) = val$, então $mlevel(\varphi, \chi_1, s0) = mlevel(\varphi, \chi_2, s1) = val$
		\item Se $mlevel(\varphi, \pos{\agent}\chi_1, s) = val$ ou $mlevel(\varphi, \nec{\agent}\chi_1, s) = val$, então $pol(\varphi, \chi_1, s0) = val+1$
		\item Se $mlevel(\varphi, \neg\chi_1, s) = val$, então $mlevel(\varphi, \chi_1, s0) = val$
	\end{itemize}
\end{definition}


Dizemos que o nível modal de $\chi_1$ em $\varphi$ na posição $s$ é $mlevel(\varphi, \psi, s)$.

A semântica para lógica modal proposicional é dada por estruturas de Kripke. Uma estrutura de Kripke $M$ é da forma $M = (\set{W}, \st_0, \set{R}_1, \ldots, \set{R}_{|\Agents|}, \pi)$, onde % TODO Cita
$\set{W}$ é um conjunto de mundos possíveis, $\st_0 \in \set{W}$, $\pi:\set{W} \times \set{P} \rightarrow \{\ctrue, \cfalse\}$, $\set{R}_\agent \subseteq \set{W} \times \set{W}$ para todo $\agent \in \Agents$. Dizemos que uma fórmula $\varphi$ é satisfatível na lógica modal K sob o modelo $M$ no mundo $\st$ se, e somente se, $\langle M,w \rangle \models \varphi$.

\begin{itemize}
	\item $\langle M,\st \rangle \models \varphi$, se e somente se $\varphi \in \set{P}$ e $\pi(\st, \varphi) = \ctrue$
	\item $\langle M,\st \rangle \models \neg\varphi$, se e somente se $\langle M,\st \rangle \not\models \varphi$
	\item $\langle M,\st \rangle \models (\varphi \land \psi)$, se e somente se $\langle M,\st \rangle \models \varphi$ e $\langle M,\st \rangle \models \psi $
	\item $\langle M,\st \rangle \models (\varphi \lor \psi)$, se e somente se $\langle M,\st \rangle \models \varphi$ ou $\langle M,\st \rangle \models \psi $
	\item $\langle M,\st \rangle \models (\varphi \then \psi)$, se e somente se $\langle M,\st \rangle \not\models \varphi$ ou $\langle M,\st \rangle \models \psi $
	
	\item $\langle M,\st \rangle \models \pos{\agent}\varphi$, se e somente se $\exists\st', (\st, \st') \in \set{R}_\agent, \langle M,\st \rangle' \models \varphi$
	\item $\langle M,\st \rangle \models \nec{\agent}\varphi$, se e somente se $\forall\st', (\st, \st') \in \set{R}_\agent, \langle M,\st \rangle' \models \varphi$
	
\end{itemize}

Uma fórmula $\varphi$ é localmente satisfatível se existe um modelo $M$ e um mundo $\st$ tal que $\langle M,\st \rangle \models \varphi$. Uma formula $\varphi$ é globalmente satisfatível se existe um modelo $M$ tal que para todo $\st \in \set{W}$ temos que $\langle M,\st \rangle \models \varphi$. Escrevemos $M \models \varphi$ se, e somente se, $\langle M, \st_0 \rangle \models \varphi$

Podemos reduzir o problema de satisfatibilidade global ao problema de satisfatibilidade local com a extensão pelo operador universal $\nec{*}$. Seja $M = (\set{W}, \set{R}_1, \ldots, \set{R}_{|\Agents|}, \pi)$, $\langle M,\st \rangle \models \nec{*}\varphi$ se, e somente se, para todo $\st' \in \set{W}$, $\langle M,\st \rangle \models \varphi$.

% satisfatibilidade local(sem e com restricoes globais)

\begin{definition}
	Uma fórmula está na forma normal negada caso seja formada somente por simbolos proposicionais, $\neg$, $\land$, $\lor$, $\nec{\agent}$ e $\pos{\agent}$ para $\agent \in \set{A}$, e a negação só é aplicada a simbolos proposicionais.
\end{definition}

É importante ressaltar $\varphi \in \set{FBF}$ que não está na forma normal negada pode ser reescrita em $\psi \in \set{FBF}$ na forma normal negada com sua satisfatibilidade preservada. Isto é, para todo $\langle M, w \rangle$, $\langle M, w \rangle \models \varphi$ se, e somente se, $\langle M, w \rangle \models \psi$ \cite{alguemqueprovaisso}. % TODO

\section{Forma Normal em Camadas}
O provador utiliza uma outra linguagem chamada de Forma Normal Separada em Níveis Modais($SNF_{ml}$).

\begin{definition}
	Uma fórmula em $SNF$ é uma conjunção de cláusulas. Para $ml \in \mathbf{N} \cup \{*\}$ e $l_1, l_2 \in \set{LP}$. Cada cláusula está em um dos três formatos:
	\begin{itemize}
		\item $ml: c$, onde $c$ é uma cláusula proposicional
		\item $ml: l_1 \rightarrow \nec{a}l_2$
		\item $ml: l_1 \rightarrow \pos{a}l_2$
	\end{itemize}
\end{definition}

A satisfatibilidade de $SNF_{ml}$ é definida pela satisfatibilidade de $K$. Sejam $ml: \varphi$ e $ml: \psi$ cláusulas $SNF$ e $M$ um modelo na lógica $K$.

\begin{itemize}
	\item $M \models *: \varphi$ se, e somente se, $M \models \nec{*}\varphi$.
	\item $M \models (ml: \varphi) \land (ml: \psi)$ se, e somente se, $M \models ml: \varphi$ e $M \models ml: \psi$.
	\item $M \models ml: \varphi$ se, e somente se, para todo $\st$ tal que $depth(\st) = ml$, temos $\langle M, \st \rangle \models ml: \varphi$.
\end{itemize}

\cite{nalontocl} descreve uma função de tradução de $K$ para $SNF_{ml}$ bem como prova que a tradução de uma fórmula mantém sua satisfatibilidade.

\section{Regras de inferência}
\cite{nalontocl} introduziu o cálculo dedutivo baseado em resolução $RES_{ml}$ para lógica $K_n$. Para lidar com $*$ as regras de inferência são simplificadas com o uso de uma função parcial de unificação $\sigma \colon P(\mathbf{N} \cup \{*\}) \mapsto \mathbf{N} \cup \{*\}$, tal que $\sigma(\{ml, *\}) = ml$, $\sigma(\{ml\}) = ml$. As regras de inferência de $REM_{ml}$ são apresentadas na Figura~\ref{figura:regrasinf}, onde $*-1=*$ e $m$ pode ser 0. Essas regras só valem se o resultado da unificação for definido. \cite{nalontocl} prova correção e corretude de todas as regras.

\begin{figure*}[t]
	\[
	\begin{array}{c}
	\begin{array}{ccc}

	\begin{array}{c}
	ml: (D \lor l) \\
	ml': (D' \lor \neg l) \\
	\hline
	\sigma(\{ml, ml'\}): D \lor D' \\
	\end{array}
	&

	\begin{array}{c}
	ml: (l_1 \rightarrow \nec{\st} l) \\
	ml': (l_2 \rightarrow \pos{\st} \neg l) \\
	\hline
	\sigma(\{ml, ml'\}): \neg l_1 \lor \neg l_2 \\
	\end{array}
	&

	\begin{array}{c}
	ml_1: (l_1' \rightarrow \nec{\st} l_1) \\
	ml_2: (l_2' \rightarrow \nec{\st} \neg l_1) \\
	ml_3: (l_3' \rightarrow \pos{\st} l_2) \\
	\hline
	\sigma(\{ml_1, ml_2, ml_3\}): \neg l_1' \lor \neg l_2' \lor \neg l_3' \\
	\end{array}
	
	
	\\ \end{array} \\

	\begin{array}{cc}
	\begin{array}{c}
	ml_1: (l_1' \rightarrow \nec{\st} l_1) \\
	\vdots \\
	ml_m: (l_m' \rightarrow \nec{\st} \neg l_m) \\
	ml_{m+1}: (l' \rightarrow \pos{\st} \neg l) \\
	ml_{m+2}: (l_1 \lor \ldots \lor \ldots \lor l_m \lor l) \\
	\hline
	ml: \neg l_1' \lor \ldots \lor \neg l_m' \lor \neg l' \\
	ml = \sigma(\{ml_1, \ldots, ml_{m+1}, ml_{m+2}-1\}) \\
	\end{array}
	&
	\begin{array}{c}
	ml_1: (l_1' \rightarrow \nec{\st} l_1) \\
	\vdots \\
	ml_m: (l_m' \rightarrow \nec{\st} \neg l_m) \\
	ml_{m+1}: (l' \rightarrow \pos{\st} l) \\
	ml_{m+2}: (l_1 \lor \ldots \lor \ldots \lor l_m) \\
	\hline
	ml: \neg l_1' \lor \ldots \lor \neg l_m' \lor \neg l' \\
	ml = \sigma(\{ml_1, \ldots, ml_{m+1}, ml_{m+2}-1\}) \\
	\end{array}
	
	\\ \end{array} \\
	\end{array} 
	\]
	\caption{Regras de inferência}
	\label{figura:regrasinf}
\end{figure*}

\section{Algoritmo}

\subsection{Seleção de cláusula}
A ordem nas quais as cláusulas são escolhidas para aplicação de regras de inferência podem mudar drásticamente o tamanho da prova. %TODO exemplo pequeno e "grande"
Sendo assim, é comum o estudo de métodos de seleção para implementação de provadores de teoremas. Alguns deles são Conjunto de Suporte, Resolução Negativa e Resolução Ordenada. \cite{nalontocl} propôs extensões de Resolução Negativa e Resolução Ordenada para $RES_{ml}$ as quais descreveremos a seguir.

% \subsection{Modal}
% citar correcao e completude e terminacao do calculo
% implementacao do calculo
% loop principal do algoritmo
% como funciona
% quais estrategias eh baseado
% quais opcoes existentes que sao relevantes pro meu prob(escolha de clausulas e selecao de literais)

\section{Grafo}
Um grafo é um par um ordenado $(V, E)$, onde $E \subseteq V\times V$, chamamos $V$ o conjunto de vértices e $E$ o conjunto de arestas.

\section{Matroide}
Seja $X$ um conjunto de objetos e $I \subseteq 2^X$ o conjunto de conjuntos independentes tal que:
\begin{enumerate}
	\item $\emptyset \in I$
	\item $A \in I, B \subseteq A \implies B \in I$
	\item Axioma do troco, $A \in I, B \in I, |B| > |A| \implies \exists x \in B \setminus A : A \cup \{x\} \in I$
	\item Se $A \subseteq X$ e $I$ e $I'$ são conjuntos independentes maximais de A então $|I| = |I'|$
\end{enumerate}
Então $(X, I)$ é um matroide. O problema combinatório associado a ele é: Dada um função de peso $w(e) \geq 0 ~\forall e \in X$, encontre um subconjunto independente com maior soma de pesos possível.
